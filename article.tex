\documentclass[12pt,a4paper]{article}

% --- Packages ---
\usepackage[utf8]{inputenc}
\usepackage{amsmath, amssymb, amsthm}
\usepackage{hyperref} % clickable links
\usepackage{graphicx} % figures if needed
\usepackage{enumitem} % nicer lists
\usepackage{geometry} % margins
\geometry{margin=1in}

% --- For code blocks ---
\usepackage{listings}
\usepackage{xcolor}

\lstset{
    language=Python,
    basicstyle=\ttfamily\small,
    keywordstyle=\color{blue},
    stringstyle=\color{red},
    commentstyle=\color{gray},
    breaklines=true,
    frame=single,
    numbers=left,
    numberstyle=\tiny,
    stepnumber=1,
    tabsize=4,
    showstringspaces=false
}

% --- Theorem Environments ---
\newtheorem{definition}{Definition}[section]
\newtheorem{theorem}{Theorem}[section]
\newtheorem{proposition}{Proposition}[section]
\newtheorem{example}{Example}[section]

% --- Title ---
\title{Formal Epistemology Meets Computation:\\
       Modeling Inferable Beliefs with Python}
\author{Ali Zolfagharian\thanks{Email: \texttt{ali.zolfagharian@gmail.com}}}
\date{\today}

\begin{document}

\maketitle

% --- Abstract ---
\begin{abstract}
This paper develops a framework for formal epistemology using computational tools.
We present methods to model epistemic states, inferable belief sets, and degrees of acceptance
for propositions. All computations are implemented in Python, and the library is openly
available for reproducibility. 
\end{abstract}

% --- Keywords ---
\textbf{Keywords:} Formal epistemology, inferable beliefs, Python, computational philosophy

% --- Introduction ---
\section{Introduction}
Formal epistemology provides tools to analyze belief, knowledge, and uncertainty.  
In this work, we integrate computational methods with formal epistemology to allow
practical simulations and reasoning. 

We also release a Python library implementing these methods, enabling reproducible experiments:
\begin{itemize}
    \item GitHub: \url{https://github.com/yourusername/yourlibrary}
    \item Zenodo DOI (optional): \url{https://doi.org/10.xxxx/zenodo.xxxx}
\end{itemize}

% --- Preliminaries / Formal Definitions ---
\section{Formal Definitions}
\begin{definition}[Epistemic Space]
An epistemic space is defined as a tuple $(\mathcal{P}, \mathcal{W}, m)$ where
\begin{itemize}
    \item $\mathcal{P}$ is a set of propositions,
    \item $\mathcal{W}$ is a set of possible worlds,
    \item $m: \mathcal{P} \to [0,1]$ assigns a mass (degree of belief) to each proposition.
\end{itemize}
\end{definition}

\begin{definition}[Inferable Base]
A set of propositions $B \subseteq \mathcal{P}$ is an inferable base if the intersection of
worlds where all propositions in $B$ hold is non-empty, and $B$ is maximal under this property.
\end{definition}

\begin{example}[Python Implementation]
We compute inferable bases using the following Python snippet:

\begin{lstlisting}
es = EpistemicSpace(cf, mass=masses)
inferable_bases = es.get_inferable_base()
\end{lstlisting}
\end{example}

% --- Computational Examples ---
\section{Computational Experiments}
We illustrate computations of inferable bases and degree of acceptance with the Python library:

\begin{enumerate}
    \item Define masses for propositions
    \item Compute endorsed focal subsets
    \item Compute inferable bases
    \item Compute degrees of acceptance using minimal grounds
\end{enumerate}

\begin{example}[Degree of Acceptance]
For proposition $P_{13}$:
\[
A(P_{13}) = 1 - \prod_{j} (1 - s_j)
\]
where $s_j$ are the strengths of minimal grounds in the inferable base.
\end{example}

% --- Discussion ---
\section{Discussion}
Our computational framework allows philosophers to experiment with formal models of
epistemic states and verify theoretical results. Linking code and theory ensures
reproducibility and transparency in formal epistemology research.

% --- Conclusion ---
\section{Conclusion}
We presented a Python-based approach to formal epistemology. Preprints can be shared
open-access (PhilArchive/arXiv), while peer-reviewed publication ensures academic validation.
The library is publicly available to allow replication and further development.

% --- References ---
\begin{thebibliography}{9}
\bibitem{philarchive} PhilArchive, \url{https://philarchive.org/}
\bibitem{arxiv} arXiv, \url{https://arxiv.org/}
\bibitem{fenstad} Fenstad, J. E., et al. \textit{Formal Epistemology.} Synthese, 2000.
\bibitem{yourlib} Zolfagharian, A. Python library for Epistemic Space. \url{https://github.com/yourusername/yourlibrary}
\end{thebibliography}

\end{document}
